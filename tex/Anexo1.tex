\chapter{Documentação Original}%
\small\begin{verbatim}
% -*- mode: LaTeX; coding: utf-8; -*-
%%%%%%%%%%%%%%%%%%%%%%%%%%%%%%%%%%%%%%%%%%%%%%%%%%%%%%%%%%%%%%%%%%%%%%%%%%%%%%%
%% File    : unb-cic.cls (LaTeX2e class file)
%% Authors : Flávio Maico Vaz da Costa
%%              (based on previous versions by José Carlos L. Ralha)
%% Version : 0.93
%% Updates : 0.5  [??/11/2004] - Initial release. don't remember the day.
%%         : 0.75 [04/04/2005] - Fixed font problems, UnB logo
%%                               resolution, keywords and palavras-chave
%%                               hyphenation and generation problems,
%%                               and a few other problems.
%%         : 0.8  [08/01/2006] - Corrigido o problema causado por
%%                               bancas com quatro membros. O quarto
%%                               membro agora é OPCIONAL.
%%                               Foi criado um novo comando chamado
%%                               bibliografia. Esse comando tem dois
%%                               argumentos onde o primeiro especifica
%%                               o nome do arquivo de referencias
%%                               bibliograficas e o segundo argumento
%%                               especifica o formato. Como efeito
%%                               colateral, as referências aparecem no
%%                               sumário.
%%         : 0.9 [02/03/2008]  - Reformulação total, com nova estrutura
%%                               de opções, comandos e ambientes, adequação
%%                               do logo da UnB às normas da universidade,
%%                               inúmeras melhorias tipográficas,
%%                               aprimoramento da integração com hyperref,
%%                               melhor tratamento de erros nos comandos,
%%                               documentação e limpeza do código da classe.
%%         : 0.91 [10/05/2008] - Suporte ao XeLaTeX, aprimorado suporte para
%%                               glossaries.sty, novos comandos \capa, \CDU
%%                               e \subtitle, ajustes de margem para opções
%%                               hyperref/impressao.
%%         : 0.92 [26/05/2008] - Melhora do ambiente {definition}, suporte
%%                               a hypcap, novos comandos \fontelogo e
%%                               \slashedzero, suporte [10pt, 11pt, 12pt].
%%                               Corrigido bug de seções de apêndice quando
%%                               usando \hypersetup{bookmarksnumbered=true}.
%%         : 0.93 [09/06/2008] - Correção na contagem de páginas, valores
%%                               load e config para opção hyperref, comandos
%%                               \ifhyperref e \SetTableFigures, melhor
%%                               formatação do quadrado CIP. 
%%
%% Definição de classe para dissertações do departamento de
%% Ciência da Computação (CIC) da Universidade de Brasília.
%%
%%%%%%%%%%%%%%%%%%%%%%%%%%%%%%%%%%%%%%%%%%%%%%%%%%%%%%%%%%%%%%%%%%%%%%%%%%%%%%%
%% PRÉ-REQUISITOS (usando LaTeX ou XeLaTeX):
%%
%% * Uma distribuição LaTeX2e compatível
%%   - em Windows, recomenda-se MiKTeX: http://miktex.org/
%% * Logomarca da UnB (arquivos positivo_cor.eps e contorno_preto.eps)
%%   - disponível na página: http://www.unb.br/unb/marca/index.php
%% * Pacotes obrigatórios
%%   - xkeyval, graphicx, boites, setspace,
%%     geometry, atbegshi, hyperref
%%
%% PRÉ-REQUISITOS (usando XeLaTeX):
%% * Fonte "Arial" instalada no sistema operacional
%% * Pacotes obrigatórios
%%   - fontspec, xltxtra (carregar no próprio documento)
%%
%% PRÉ-REQUISITOS (usando LaTeX):
%% * Fonte ua1 (URW Arial A030, PostScript Type-1)
%%   - disponível no CTAN: http://www.ctan.org/tex-archive/fonts/urw/arial/
%%   - no MiKTeX, basta instalar pacote "arial" via MiKTeX Package Manager
%% * Pacotes obrigatórios
%%   - relsize, fixltx2e
%% * Pacotes opcionais
%%   - MinionPro (fonte proprietária)
%%
%%%%%%%%%%%%%%%%%%%%%%%%%%%%%%%%%%%%%%%%%%%%%%%%%%%%%%%%%%%%%%%%%%%%%%%%%%%%%%%
%% PRINCIPAIS OPÇÕES DISPONÍVEIS:
%%
%% licenciatura - O trabalho é do curso de Licenciatura (Graduação).
%% bacharelado  - O trabalho é do curso de Bacharelado (Graduação).
%% mestrado     - O trabalho é de Mestrado.
%%
%% draft - Gera o documento em modo de rascunho: não exibe as imagens e
%%         modifica o espaçamento entre linhas.
%% final - Gera o documento em modo final, não-rascunho (padrão).
%%
%% impressao       - Gera a versão para impressão, sem hipertexto.
%% hyperref=load   - Carrega e configura o pacote hyperref para gerar
%%                   a versão hipertexto para exibição em tela (padrão).
%% hyperref=config - Configura o pacote hyperref para gerar
%%                   a versão hipertexto para exibição em tela, mas o
%%                   pacote deve ser carregado no próprio documento.
%%                   Leia a seção PROBLEMAS CONHECIDOS para recomendação
%%                   de uso desta opção.
%%
%% 10pt - Define tamanho da fonte do texto.
%% 11pt - Define tamanho da fonte do texto (padrão no modo draft).
%% 12pt - Define tamanho da fonte do texto (padrão no modo final).
%%
%% singlespacing  - Define espaçamento simples entre linhas (padrão no modo
%%                  final).
%% onehalfspacing - Define espaçamento de um e meio entre linhas (padrão no
%%                  modo draft).
%% doublespacing  - Define espaçamento duplo entre linhas.
%% baselineskip   - Utiliza o comando \baselineskip para definir o espaçamento
%%                  de linhas. Se não for infomada, o espaçamento é definido 
%%                  pelo pacote "setspace" (recomendável, ou seja, só use esta
%%                  opção se tiver algum motivo para não usar o setspace).
%%                  Pode ser usada em conjunto com uma das três opções acima.
%%
%% prestyle=<val>  - Especifica o estilo das páginas iniciais (agradecimentos,
%%                   resumo, sumário, etc.). Valores típicos são "plain"
%%                   (padrão) ou "empty".
%% textstyle=<val> - Especifica o estilo das páginas dos elementos textuais e
%%                   pós-textuais. Valores típicos são "plain" (padrão) ou
%%                   "fancy" (este último requer o pacote {fancyhdr}).
%% chapstyle=<val> - Especifica o estilo da primeira página de cada capítulo.
%%                   Valores típicos são "plain" (padrão) ou "empty".
%%
%%%%%%%%%%%%%%%%%%%%%%%%%%%%%%%%%%%%%%%%%%%%%%%%%%%%%%%%%%%%%%%%%%%%%%%%%%%%%%%
%% CRIAÇÃO DA MONOGRAFIA:
%%
%% A folha de rosto e folha de aprovação, por padronização estética,
%% são sempre formatados como "onehalfspacing" independente das opções
%% escolhidas. A seção de Referências é sempre "singlespacing". Citações
%% {quote}, {quotation} e versos {verse} são automaticamente "singlespacing" e
%% com letra menor.
%%
%% Os comandos \textlinf (lining figures) e \texttabf (tabular figures)
%% permitem selecionar, respectivamente, números "maiúsculos" (sim, isso
%% existe!) e números monoespaçados - o que só faz sentido para fontes 
%% que conhenham números maiúsculos x minúsculos e monoespaçados x
%% proporcionais (Minion Pro, Myriad Pro, Didot, Apple Chancery...).
%% Essas opção foi implementada apenas para a fonte Adobe Minion Pro no LaTeX,
%% mas funciona para qualquer fonte que suporte este recurso no XeLaTeX -
%% veja a documentação do pacote {fontspec} para mais informações.
%% Use \textlinf quando os números acompanharem texto em letras maiúsculas ou
%% em versalete (small-caps). Use \texttabf em tabelas ou outros lugares onde
%% for desejável que os números fiquem alinhados em colunas. Como alternativa
%% aos comandos, pode-se usar os ambientes {linfig} e {tabfig}.
%% O comando \SetTableFigures pode ser usado para redefinir um ambiente de
%% modo que os números contidos nele sejam automaticamente formatados
%% maiúsculos e monoespaçados, por exemplo: 
%%
%%   \SetTableFigures{tabular}
%%   \SetTableFigures{tabularx}
%%   \SetTableFigures{longtable}
%%
%% A capa, quando presente, não é contada na numeração das páginas.
%% As páginas iniciais, a partir da folha de rosto, são consideradas na 
%% contagem de páginas, mas não são numeradas.
%% As páginas seguintes (lista de figuras, etc.) por padrão são numeradas com
%% algarismos romanos e aparecem no sumário. Entretanto, se prestyle=empty,
%% não recebem numeração nem aparecem no sumário. 
%% As páginas do conteúdo (da introdução em diante) são numeradas com 
%% algarismos arábicos (13, 14, 15...) e começam a partir de 1.
%%
%% O sumário, por razões óbvias, deve colocado antes de todas as páginas
%% que são mencionadas nele.
%%
%% O trabalho é dividido em elementos pré-textuais, textuais e pós-textuais.
%% Estes elementos (alguns são opcionais, converse com seu orientador)
%% são dispostos preferencialmente na seguinte ordem:
%%    ELEMENTO                COMANDOS MAIS IMPORTANTES
%%       Capa                 \capa
%%    Elementos pré-textuais  \pretextual
%%       Folha de rosto       \folharosto
%%       Ficha catalográfica  \cip (impressa no verso da folha de rosto)
%%       Errata               %deve ser criada como documento avulso
%%       Folha de aprovação   \folhaaprovacao
%%       Dedicatória          \begin{dedicatoria}...\end{dedicatoria}
%%       Agradecimentos       \begin{agradecimentos}...\end{agradecimentos}
%%       Epígrafe             %aluno-poeta, esse recurso não está disponível
%%       Resumo               \begin{resumo}...\end{resumo}
%%       Abstract             \begin{abstract}...\end{abstract}
%%       Lista de Figuras     \figuras
%%       Lista de Tabelas     \tabelas
%%       Lista de Siglas      \printglossary[type=acronym] %usando pacote {glossaries}
%%       Sumário              \sumario
%%    Elementos textuais      \textual
%%       Introdução           \chapter{Introdu\c{c}\~ao} ...
%%       Desenvolvimento      \chapter{Nome do capitulo} \section{Nome da secao} ...
%%       Conclusão            \chapter{Conclus\~ao} ...
%%    Elementos pós-textuais  \postextual
%%       Referências          \bibliographystyle{...} \bibliography %usar BibTeX
%%       Glossário            \printglossary %usando pacote {glossaries}
%%       Apêndices            \apendices \chapter{Nome do apendice} ...
%%       Anexos               \anexos \chapter{Nome do anexo} ...
%%       Índices              \printindex %usando pacote {makeidx}
%%
%% O comando \maketitle equivale a chamar:
%% \pretextual\folharosto\cip\folhaaprovacao
%%
%% A opção "impressao", além de ajustar as margens para melhor
%% encadernação, também faz \maketitle chamar internamente o
%% comando \capa.
%%
%% Caso esteja preparando a versão para visualização em tela
%% (hipertexto), o documento pode ser configurado com o comando
%% \hypersetup a ser chamado no preâmbulo - mais detalhes no
%% manual do pacote hyperref. Algumas opções interessantes,
%% com valores de exemplo, são:
%%   bookmarksnumbered=true,
%%	   (itens das seções exibidas no menu do PDF são numerados)
%%   linktocpage=true,
%%	   (links do sumário nos números de página, não nos nomes de seções)
%%   colorlinks=true,
%%	   (indica links pela cor do texto, não por retângulo ao redor)
%%   linkcolor=red!80!black,
%%	   (cor de links em geral, veja pacote xcolor para opções de cor)
%%   urlcolor=blue,
%%	   (cor de links para URLs - comando \url{http://...})
%%   citecolor=teal,
%%	   (cor de links para referências citadas no texto)
%%   pdfstartview=FitH
%%	   (PDF exibido inicialmente com zoom para largura da página)
%%
%% O comando \listas equivale a chamar:
%% \figuras\tabelas
%%
%% O alinhamento e formatação dos títulos é controlado pelos comandos
%% \pretextual, \textual e \postextual, de modo que em cada parte do trabalho
%% os títulos são apresentados de maneira diferente. Se quiser mudar o
%% alinhamento do título numa das partes do trabalho, logo após a chamada ao
%% comando acima, redefina o comando \chapteralign para aplicar o alinhamento
%% desejado. Se quiser fazer outras mudanças no formato (por exemplo, colocar
%% os títulos em versalete (small-caps), será necessário redefinir o comando
%% \chapterformat (veja exemplo no código da classe).
%%
%% Redefina o comando \fontelogo para mudar o nome da fonte a ser utilizada
%% no logo da UnB, sendo que o padrão é "Arial" (XeLaTeX) ou "ua1" (LaTeX).
%% Deve-se trocar por uma fonte parecida (como a Helvetica, "phv" no LaTeX)
%% apenas se não for possível a instalação da Arial.
%%
%% A classe adicionalmente provê o ambiente {definition}, que exibe
%% definições (conceituais) numeradas e formatadas com borda.
%%
%% Alguns dos comandos a serem chamados no preâmbulo (parte do código antes
%% da chamada a \begin{document}) são:
%% \title[] - Título do trabalho.
%% \subtitle[] - Subtítulo do trabalho (se houver).
%% \palavraschave[] - Lista de palavras-chave em português.
%% \keywords[] - Lista de palavras-chave em inglês.
%%   Os parâmetros opcionais definem como o valor será apresentado na página
%%   da CIP e nas propriedades do PDF, já que estas não podem conter macros
%%   (ex.: \title[Um estudo sobre PI elevado a PI]{Um estudo sobre $\pi^\pi$}).
%% \diamesano - Dia mês e ano.
%% \autor - Nome e sobrenome do autor.
%% \coautor - Nome e sobrenome do co-autor (se houver).
%% \coordenador[a] - Coordenador(a) do curso.
%% \orientador[a] - Orientador(a) do trabalho.
%% \coorientador[a] - Coorientador(a) do trabalho (se houver).
%% \membrobanca - Adiciona membros à banca (além do orientador e coorientador).
%% \CDU - Classificação Decimal Universal (por padrão é 004).
%%   Valores comuns para a área (conforme http://www.udcc.org/) são:
%%    004 Ciência da Computação e Tecnologia
%%    004.2	Arquitetura de computadores
%%    004.3	Hardware
%%    004.4	Software
%%    004.5	Interação homem-máquina
%%    004.6	Dados
%%    004.7	Comunicação de computadores
%%    004.8	Inteligência artificial
%%    004.9	Técnicas de informática orientadas a aplicação
%%
%%%%%%%%%%%%%%%%%%%%%%%%%%%%%%%%%%%%%%%%%%%%%%%%%%%%%%%%%%%%%%%%%%%%%%%%%%%%%%%
%% PROBLEMAS CONHECIDOS:
%%
%% * O pacote hyperref, embora muito útil para gerar um PDF para leitura em
%%   tela, tem inúmeros problemas de compatibilidade com outros pacotes -
%%   alguns só funcionam corretamente se forem carregados antes do hyperref.
%%   A opção [hyperref] (ou seu sinônimo [hyperref=load]) carregam o pacote
%%   na própria classe, o que impossibilita que algum pacote no documento do
%%   aluno seja carregado antes do hyperref. Neste caso, usar a opção
%%   [hyperref=config] junto com o comando \ifhyperref para carregar o
%%   hyperref apenas quando estiver sendo gerada uma versão para leitura em
%%   tela, como no seguinte exemplo:
%%
%%     %%% não carrega nameref com opção [impressao] 
%%     \ifhyperref{\usepackage{nameref}}{}
%%     %%% varioref só funciona se carregado entre nameref e hyperref
%%     \usepackage{varioref}
%%     %%% não carrega hyperref com opção [impressao] 
%%     \ifhyperref{\usepackage{hyperref}}{}
%%
%% * Recomenda-se o uso do XeTeX ou pdfTeX, já que o dvips (+ps2pdf)
%%   não suporta adequadamente links com quebra de linha. Quem deseja usar os
%%   recursos gráficos do PSTricks pode experimentar o pacote TikZ como
%%   alternativa que não depende de diretivas PostScript.
%% * O suporte para o modo matemático ainda não está completo no XeTeX. Se seu
%%   trabalho depende significativamente de fórmulas e equações, use o LaTeX
%%   (ou XeLaTeX com \usepackage[no-math]{fontspec} e uma fonte matemática do
%%   LaTeX que combine com a fonte escolhida para o texto).
%%
%%%%%%%%%%%%%%%%%%%%%%%%%%%%%%%%%%%%%%%%%%%%%%%%%%%%%%%%%%%%%%%%%%%%%%%%%%%%%%%
%% ORIENTAÇÕES TIPOGRÁFICAS:
%%
%% * Atenção aos diferentes tipos de traço:
%%      - O hífen (-) liga prefixos, sufixos, partículas e pronomes oblíquos
%%        (ex.: recordar-se, fazê-lo, co-habitar).
%%      - A meia-risca (--) indica intervalos numéricos e liga palavras
%%        independentes (ex.: páginas 1--5, viagem Londres--Estocolmo, São João
%%        del-Rei--MG).
%%      - O travessão (---) é usado em diálogos, quando o interlocutor muda,
%%        e pode substituir dois pontos, parênteses ou vírgulas em apostos.
%%   Antes e depois de travessão emprega-se espaço, nos demais casos não há 
%%   espaço.
%% * O ponto final marca o fim de uma sentença. Se o ponto for usado com outra
%%   finalidade, como numa abreviatura, usar \ (ou ~ se não quiser permitir 
%%   quebra de linha) para indicar ao TeX que não se trata de fim de sentença
%%   (ex.: Dr.\ Spock).
%% * Embora o padrão das normas ABNT (que, com suas "receitinhas de bolo", são
%%   bastante rudimentares quanto à tipografia) seja espaçamento duplo entre 
%%   linhas, o espaçamento deve ser escolhido de acordo com o tipo e tamanho
%%   das letras.
%%   A regra a ser seguida é: quanto maiores as linhas (quanto mais palavras
%%   por linha), maior deve ser o espaçamento entre elas. Muitas vezes o
%%   espaçamento simples é suficiente numa fonte bem elaborada, mas se o
%%   tamanho médio da linha for superior a 90 caracteres, pode-se aumentar a
%%   legibilidade definindo um modesto incremento de espaçamento (algo entre
%%   \SetSinglespace{1.1} e \SetSinglespace{1.15}).
%% * Como um trabalho acadêmico tem margens relativamente estreitas, para
%%   evitar que as linhas sejam muito longas, em modo final esta classe utiliza
%%   tamanho 12pt.
%%   Deve haver interesse nas opções 10pt ou 11pt apenas se estiver sendo
%%   utilizada uma fonte particularmente grande (por exemplo, que produza uma
%%   média inferior a 80 caracteres por linha a 12pt).
%% * Na tipografia brasileira e portuguesa é costume recuar (indentar) a
%%   primeira linha de cada parágrafo. Como esse recuo serve para destacar
%%   visualmente um parágrafo de seu anterior, o LaTeX não faz o recuo no
%%   primeiro parágrafo sob um título. Quem quiser indentar também o primeiro
%%   parágrafo só para "manter o costume" deve adicionar o pacote {indentfirst}
%%   ao seu documento.
%% * Citações com até três linhas ficam no próprio texto, marcadas por "aspas".
%%   Se o texto original contiver "aspas", trocá-las por 'aspas simples'.
%%   Se a citação tiver mais de três linhas e apenas um parágrafo, usar o
%%   ambiente {quote}. Se a citação tiver mais de um parágrafo, usar
%%   {quotation}. Nestes dois casos, não se deve circundar sua citação por
%%   aspas nem alterar as aspas do texto original.
%%   Se a citação começar com letra maiúscula, o texto imediatamente anterior
%%   deve terminar em dois pontos (:).
%% * Números ordinais em inglês são sucedidos por "st", "nd", "rd" ou "th",
%%   de acordo com o seguinte algoritmo:
%%      Se o número termina em 1 mas não termina em 11, use "st".
%%      Se o número termina em 2 mas não termina em 12, use "nd".
%%      Se o número termina em 3 mas não termina em 13, use "rd".
%%      Em quaisquer outros casos, use "th".
%%   Esse indicador de ordinal deve, preferencialmente, ser colocado como texto
%%   superscrito. Usar 112\textsuperscript{th} e não 112$^{th}$. Se não quiser
%%   se preocupar com essas regras, use o pacote {engord} com o comando
%%   \engordnumber{112}.
%% * Para indicar elipse (reticências), use \textellipsis,
%%   usar ... (ponto ponto ponto) não é a mesma coisa.
%%
%%%%%%%%%%%%%%%%%%%%%%%%%%%%%%%%%%%%%%%%%%%%%%%%%%%%%%%%%%%%%%%%%%%%%%%%%%%%%%%
%% FIM DA DOCUMENTAÇÃO
%%%%%%%%%%%%%%%%%%%%%%%%%%%%%%%%%%%%%%%%%%%%%%%%%%%%%%%%%%%%%%%%%%%%%%%%%%%%%%%
\end{verbatim}