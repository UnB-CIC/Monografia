Neste capítulo, será descrito o aplicativo móvel \appName, assim como todas as suas funcionalidades, telas e aplicabilidade.

O aplicativo móvel foi desenvolvido com o intuito de ensinar computação para qualquer pessoa interessada de forma mais descontraída e gamificada. O \appName\ foi dividido em dois módulos, denominados \textit{back office} e \textit{front office}. O \textit{front office} foi o módulo desenvolvido para dispositivos móveis Android e é o elemento principal da aplicação, representado o local onde o aluno irá realizar os desafios, consultar sua colocação, interagir com outros alunos, entre outros. Já o \textit{back office} é uma versão Web da aplicação cujo público alvo são professores que queiram utilizar da plataforma para ensinar os seus alunos. Os professores serão capazes de criar cursos, que são compostos por tópicos, que por si, podem possuir diversos desafios, cada um valendo uma quantidade determinável de pontos que os alunos poderão obter. Por se tratar de um aplicativo em que o conteúdo pode ser adicionado a qualquer momento por um professor que se cadastre na plataforma Web, os conhecimentos prévios para o uso não podem ser determinados como um todo, mas sim por curso, a ser determinado pelo professor em questão.

O aplicativo conta com um curso já pré-cadastrado que pode ser acessado pela tela inicial por meio da denominação "Plataforma global". TODO\dots

As próximas seções mostram as telas e funcionalidades do \appName\, módulo \textit{front office} e \textit{back office} em mais detalhes.

\section{Front Office}

Versão \textit{mobile} do \appName, possui as seguintes características:

\begin{itemize}
    \item \textbf{Público-alvo}: Estudantes de computação que estejam interessados em aprender de forma gamificada ou alunos de professores que utilizam a plataforma como elemento de aprendizagem de seus respectivos cursos;
    \item \textbf{Plataforma}: Dispositivos móveis, principalmente \textit{smartphones} Android;
    \item \textbf{Objetivo geral}: Aprendizagem gamificada de computação.
\end{itemize}
